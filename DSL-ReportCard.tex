\documentclass[11pt]{article}

\usepackage{mathptmx,amssymb,amsmath}
\usepackage{fullpage}
\usepackage{enumerate}
\usepackage{fancyvrb}


%%%
%%% Formatting details
%%%
\sloppy
\sloppypar
\widowpenalty=0
\clubpenalty=0
\displaywidowpenalty=0
\raggedbottom
\pagestyle{plain}


% \topskip0pt
% \parskip0pt
% \partopsep0pt

\DefineVerbatimEnvironment{program}{Verbatim}
  {baselinestretch=1.0,xleftmargin=5mm,fontsize=\small,samepage=true}

\def\denseitems{
    \itemsep1pt plus1pt minus1pt
    \parsep0pt plus0pt
    \parskip0pt\topsep0pt}


%%%
%%% A few macros
%%%
\newcommand{\RevisionNeeded}{\bigskip\noindent\fbox{
\textbf{Be prepared to revise this section \emph{many} times!}}}

\newcommand{\prog}[1]{{\small\texttt{#1}}}
% \newcommand{\bs}{\texttt{\symbol{92}}}



\begin{document}

\title{\textbf{DSL Name}
\\ DSL Report Card}

\author{Author \\
School of EECS \\ 
Oregon State University
}

\maketitle


\section{Introduction}
\label{sec:intro}

Provide a brief summary and some background of the domain. Why is a DSL
needed, and how would it be useful?  What notations/languages, if any, are
currently used? What are their strengths and limitations?


\section{Users}
\label{sec:users}

How is the DSL going to be used, and what specifically is it used for?  
% 
What kind of users are affected by the DSL and in which way? Who is writing
programs and who is using them? What is known or can be reasonably assumed
about the (technical/domain) background of users? (Usually, it is safe to
assume that users are domain experts.)

Specifically, what are the limitations of the DSL? What cannot be expressed?
If different kinds of users are involved, what are there limitations of for
the different user groups?



\section{Outcomes}
\label{sec:outcomes}

What is (are) the outcome(s) of executing a DSL program? What analyses of DSL
programs are conceivable?  Note that, in general, a DSL program can produce a
variety of results. In fact, this aspect makes the use of a DSL very attractive, and this
questions is an important part of the analysis of the DSL's purpose.

Consider, for example, a DSL for describing vacation plans (including
alternatives).  We can imagine that a vacation plan can be analyzed to yield a
set of \emph{date/time ranges} for the vacation (based on the availability of
flights, hotels, etc.), or a set of \emph{cost estimates} (based on travel and
date options). Moreover, we can imagine deriving \emph{to-do lists} from the
travel plan (making reservations, buying tickets, etc.) or \emph{entertainment
suggestions} (parks, museums, events, etc.). 

The outcomes should be made more precise in Section \ref{sec:analysis} in the
form of function signatures once the basic objects and combinators have been
identified in Sections \ref{sec:objects} and \ref{sec:comb}.



\section{Use Cases / Scenarios}
\label{sec:examples}

Describe several typical example problems or use cases that can be expressed
in the DSL. For each example, give a summary of what the example is about, and
explain how important and representative it is for the domain.

Then describe in more detail the steps involved in solving the problem (either
by hand or with existing tools). This process can be very helpful in shedding
light on what kind of basic objects, types, and combinators the DSL will have
to offer.



\section{Basic Objects}
\label{sec:objects}

What are the basic objects that are manipulated and used by the DSL? Basic
objects are those that are not composed out of other objects. 

As a general rule, the fewer basic objects one needs, the better, because the
resulting DSL design will be more concise and elegant.
%
The basic objects should be described by a set of Haskell \prog{type} and
\prog{data} definitions. Use the \prog{program} environment to show code, as
illustrated below.

\begin{program}
type Point = (Int,Int)
\end{program}
%
Show how (some (parts) of) the examples from Section \ref{sec:examples} will be
represented by values of the envisioned types. For example:

\begin{program}
home :: Point
home = (10,13)
\end{program}
%
Also, list current limitations that you expect in a future iterations to overcome.

\RevisionNeeded



\section{Operators and Combinators}
\label{sec:comb}

Identify operators that either transform objects into one another or
build more complex objects out of simpler (and ultimately basic) ones.
%
Depending on what implementation or form of embedding will be chosen,
operators may be given as constructors of data types or functions.

Combinators are higher-order functions that encode control structures of the
DSL. The function \prog{map} is a combinator that realizes a looping construct
for lists. The operations of the parser library Parsec are called \emph{parser
combinators} since parsers themselves are represented as functions.
%
The identification of the right set of combinators is a key step in the design
of the DSL.

With basic objects, operators, and combinators, you should be able to
demonstrate how the examples from Section \ref{sec:examples} can be
represented. All limitations encountered here should be classified as either:

\begin{enumerate}[(1)]\denseitems
\item Temporary
\item Fundamental
\end{enumerate}
%
Temporary limitations should be noted in this section and in  Section
\ref{sec:objects} as \emph{TO DO} items for future revisions of the design.
%
Fundamental limitations should be reported and listed in detail in Section
\ref{sec:users}.

\RevisionNeeded



\section{Interpretation and Analyses}
\label{sec:analysis}

Provide a precise description of the different outcomes of the DSL. This
information is typically provided by Haskell function signatures (that is,
function names and their types).

One kind of ``outcome'' is the result of executing a DSL program. For example,
for a scheduling DSL, an outcome would typically consist of a set of possible
schedules. But it could also be only one schedule (the best according to some
criterion).
%
However, in addition to such primary outcomes, there could be other results.
In the case of the scheduling DSL, one could imagine sending emails to
participants asking for priorities in case several optimal schedules exist.



\section{Cognitive Dimension Evaluation}
\label{sec:cogdim}

An assessment of cognitive dimension of your notation, such as closeness of
mapping, viscosity, hidden dependencies, and others, will help you with the
re-design of the DSL.

Note that it is advisable to think about the cognitive dimensions constantly
during the design of your DSL. Cognitive dimensions are, in fact, a language
design tool, and not so much intended to evaluate languages after the fact.


\section{Implementation Strategy}
\label{sec:implementation}

Discuss how the advantages and disadvantages of a deep or shallow embedding
play out in your DSL.

What dvantages would an implementation in a language workbench have? Is it
worth the effort?



\section{Find Similar/Related DSLs}
\label{sec:related}

Try to find DSLs that are similar to the one described here and compare your
DSL with those. Note that similarity can be understood as \emph{topical} as
well as \emph{technical} similarity .

\emph{Topically similar} DSLs are DSLs for the same or a closely related
domain. They have in principle the same or slightly different outcomes, but
they may be implemented quite differently. These DSLs help you refine the
design of the DSL requirements described in Sections \ref{sec:users} and
\ref{sec:outcomes}.

\emph{Technically similar} DSLs are Haskell DSLs whose types and functions are
similar to the ones used in Sections \ref{sec:objects} and \ref{sec:comb}.
They can help sharpen the DSL modeling and implementation described in those
sections.

Ideally, you can find both, topically and technically similar DSLs. Be sure to
properly cite the DSLs as references.

\bigskip\noindent\fbox{
\textbf{You will have to present these DSLs in a class presentation.}}



\section{Design Evolution}
\label{sec:evol}

As you iterate over different designs of your DSL, it is quite instructive to
document some of the old, obsolete designs, that is, show the type definitions
and function signatures, explain why this design seemed attractive at first
and then what motivated you to change it.

This part may seem like an unnecessary burden to you, but it helps you and
others to understand your current design, and it probably answers questions
that users (or reviewers) of the DSL might have about the design, because they
may have thought of your initial design also and are wondering why it has not
been adopted.



\section{Future Work}
\label{sec:future}

A speculation about what it takes to remove some of the limitations and
whether it seems worth the effort.

Moreover, what would be the concrete benefits to extend a shallow DSL into a
deep DSL? Or, would it be helpful to create an external DSL? What role could a
visual syntax or a GUI interface play?



\section*{References}

List references to similar DSLs identified in Section \ref{sec:related} and
potentially other related work.


\end{document}

